\chapter{Conclusion and Future Work} \label{ch:conclusion}

\section{Summary}

Over the course of this thesis a three dimensional ray tracer application was built in C++ to produce a RBM signal according to a variety of physical parameters. The physical parameters associated with a helicopter blade and the positioning of transmitter and receiver hardware were swept independently to build knowledge on how they affect the resulting signal. That data was then processed using time frequency analysis to produce a Doppler envelope. An envelope of each parameter step was used to determine an overall trend. Then with those trends an algorithm was derived to find the azimuth and elevation angle of the transmitter with relationship to the helicopter platform. The algorithm accurately finds the azimuth angle and elevation using the data collected in the simulation environment with little error in the azimuth estimation and less than 2\% error in elevation angle when blade pitch in 0\textdegree under a deterministic signal. With the addition of Gaussian noise the algorithm stays relatively constant up to a certain point, then both the azimuth and elevation estimations start to degrade with decreasing elevation angle and increasing noise level. For the simulation scenarios conducted, once the level of noise exceeded 10dB the algorithm started to degrade, but this level will most likely change if other simulation parameters are chosen.

\section{Future Work}
%Real world experiment needs to be conducted starting with small scale and moving towards a large scale implementation on a helicopter platform.
The first thing that needs to be done in continuing this research is to perform a real world experiment to officially evaluate the ray tracing application. To do this appropriate hardware needs to be defined in the form of a software defined radio for both the receiver and transmitter along with a software workflow such as GNU Radio. Next a small scale environment would need to be set up, with some form of rotating metal blade or thin plate rotating at a sufficient rate to produce a Doppler spread similar to those defined in the simulation. After that data is collected and analyzed a full scale experiment would need to be conducted including a real helicopter platform and an environment with large expanses of flat ground.

%Evaluate the estimation techniques on the real world collected data to see if they are still applicable.
Once the real world experiments have been conducted, and are shown to be similar to the signal produced by the simulation, the estimation techniques defined in this thesis can be evaluated on the real signal to determine if they are sufficient in practice.

%Improve propagation model to better represent the real world conditions, material, more physical effects on RF, near field problems etc.
After the physical validation the ray tracing application can be modified to include more physical phenomenon. The other improvement would be to introduce more objects into the scene model which would require modification to the tracing algorithm to include ray casting distributions in their particular direction.

%Improve computation speed by employing multithreading or computation on GPU Architecture for reduced simulation time, not necessary but would be more convenient.
Another improvement on the simulation would be to increase the speed of computation. This could be done by implementing multi-threading or, since raytracing is heavily parallelizable, a Graphics Processing Unit (GPU) based architecture. The multi-threading option would require minor code modifications but realizing the simulation on a GPU would require a full rewrite of the software to take advantage of the GPU's abilities. 

%Introduce a pitch parameter to the estimation equation based on empirical data, so that if the blades are pitched the doppler can be adjusted accordingly.
The current estimation does not take into account the physical pitch value when correcting for pitch and might not cover all the cases that pitch can effect. Therefore further investigation is needed into how the pitch affects the received signal for improved azimuth and elevation estimation.

%Because there is more information in the received envelopes besides the maximum and minimum the specific shape could be used to better estimate azimuth and elevation without the need to complete a full revolution as described in the proposed method.
The estimation methods described in this thesis rely only on the maximum and minimum values of the Doppler envelopes calculated but their shape could also be of particular interest as well. When varying the parameters of azimuth and elevation the shape of the envelope changes in a distinctive manner, by using a training algorithm the position of the transmitter might be able to be tracked in time by evaluating the change in shape. Under the defined estimation methods a full revolution of the helicopter platform is needed to resolve position ambiguity.
 
