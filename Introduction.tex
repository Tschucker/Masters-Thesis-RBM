\chapter{Introduction}

\section{Motivation}
In RF communications with a rotorcraft, such as a helicopter, the rotor blades can impart a modulation onto the received signal called Rotor Blade Modulation (RBM). This modulation is caused by the reflection of a signal off the rotating blades. The reflected signal is Doppler shifted based on where the signal is reflected along the length of the blade as well as the elevation angle between the axis of rotation and the emitter. RBM can be used in RADAR to detect and classify aircraft based on a specific signature and can degrade communication system located on the rotorcraft itself. The motivation behind this work is to use the RBM effect to locate uncooperative emitters on the ground. This is done through the use of specific knowledge of the craft itself and measuring the RBM in the received signal.

\section{Contribution}
The main contributions of this thesis are as follows. A 3D Ray Tracing software is provided to simulate the RF propagation in the presence of a helicopter rotor blade. The software provides insight into the complex interaction between the rotor blade shape and the placement of the receiver and transmitters. The second contribution is an in depth analysis of how several different variables affect the modulation on the received signal. And lastly, techniques to use the imparted rotor blade modulation to estimate the azimuth and elevation angles with respect to the location of the rotating blades.

\section{Organization}
In Chapter \ref{ch:background} we present the background and theory behind rotor blade modulation. we then formulate the basic geometry behind the RBM effect.

In Chapter \ref{ch:radio_propagation} we Introduce the radio propagation simulation in several sections. First is the RF propagation model which defines the physical effects that are simulated within the software. Next the objects of the scene model, as well as the dimensionality, are defined in software to be representative of their physical  dimensions and functionality. Lastly we describe the core ray tracing engine that traces the scene in successive frames to build the receiver output.

In Chapter \ref{ch:simulations} we present the result of the parameter swept simulations of the ray tracing software. Next the received signals from the simulation are analyzed using several DSP techniques to define the signal characteristics that will be used in the next chapter for estimation. 

In Chapter  \ref{ch:results} Those characteristics are then used to estimate the azimuth and elevation angle of the transmitter based on the given rotor and receiver parameters. Those estimates are then compared with the original position of the transmitter to determine the methods accuracy.

In Chapter \ref{ch:conclusion} the summery of the results are detailed along with the current limitations. Lastly future improvements on the work will be discussed.